\documentclass[11pt,a4paper]{report}
\usepackage[dvips]{graphicx}           
\usepackage[portuges]{babel}           
\usepackage{color}            
\usepackage[latin1]{inputenc}           
\usepackage[normalem]{ulem}
\setlength{\textwidth}{160mm}
\setlength{\textheight}{220mm}
\setlength{\topmargin}{-8pt}
\setlength{\oddsidemargin}{0mm}
\usepackage{amsmath,amsfonts}
\usepackage{enumitem}% http://ctan.org/pkg/enumitem

\begin{document}

  \begin{minipage}[b]{0.1 \linewidth}
     \includegraphics[width=.9cm]{ufs_vertical_positiva.png}
  \end{minipage}\hfill	
  \begin{minipage}[b]{0.85 \linewidth}
	       {\large U}NIVERSIDADE {\large F}EDERAL DE {\large S}ERGIPE\\
	       {\large D}EPARTAMENTO DE {\large E}STAT�STICA E {\large C}I�NCIAS {\large A}TUARIAIS\\
	       Disciplina: ESTAT0072 -- Probabilidade I\\
	       Professor: Sadraque E.F. Lucena
\end{minipage}

\vspace{0.8cm}
\begin{center} {\bf Lista de Exerc�cios 1} \end{center}
\vspace{0.8cm}

	\begin{enumerate}[label=1.\arabic*)]
		\item Suponha que o conjunto fundamental seja formado pelos inteiros positivos de 1 a 10. Sejam	$A = \{2, 3, 4\}$, $B = \{3, 4, 5\}$ e $C = \{5, 6, 7\}$. Enumere os elementos dos seguintes conjuntos:
		\begin{enumerate}
			\item $A^c \cap B$.
			\item $A^c \cup B$.
			\item $A \cap (B\cup C)^c$.
			\item $A \cup (B\cap C)^c$.
		\end{enumerate}
		
		\item Suponha que o conjunto fundamental $U$ seja formado por $U = \{x ~|~ 0 \leq x \leq 2\}$. Sejam os conjuntos $A$ e $B$ definidos da forma seguinte: $A = \{x ~|~ 1/2 < x \leq 1\}$ e $B = \{x ~| ~1/4 \leq x < 3/2\}$. Descreva os seguintes conjuntos:
		\begin{enumerate}
			\item $(A\cup B)^c$.
			\item $A\cup B^c$.
			\item $(A\cap B)^c$.
			\item $A^c\cap B$.
		\end{enumerate}
		
		\item Quais das seguintes rela��es s�o verdadeiras?
		\begin{enumerate}
			\item $(A \cup B) \cap (A \cup C) = A \cup (B \cap C)$.
			\item $A \cup B = (A \cup B^c ) \cup B$.
			\item $A^c \cap B = A \cup B$.
			\item $(A \cup B)^c \cap C = A^c \cap B^c \cap C^c$.
			\item $(A \cap B) \cap (B^c \cap C) = \emptyset$.
		\end{enumerate}
		
		\item Suponha que o conjunto fundamental seja formado por todos os pontos $(x, y)$ de coordenadas ambas inteiras, e que estejam dentro ou sobre a fronteira do quadrado limitado pelas retas $x = 0$, $y = 0$, $x = 6$ e $y = 6$. Enumere os elementos dos seguintes conjuntos:
		\begin{enumerate}
			\item $A = \{(x, y) ~|~ x^2 + y^2 \leq 6\}$.
			\item $B = \{(x, y) ~|~ y \leq x^2 \}$.
			\item $C = \{(x, y) ~|~ x \leq y^2 \}$.
			\item $B \cap C$.
			\item $(B \cup A) \cap C^c$.
		\end{enumerate}
		
		\item Empregue o diagrama de Venn para estabelecer as seguintes rela��es:
		\begin{enumerate}
			\item $A \subset B$ e $B \subset C$ implica que $A \subset C$.
			\item $A \subset B$ implica que $A = A \cap B$.
			\item $A \subset B$ implica que $B^c \subset A^c$.
			\item $A \subset B$ implica que $A \cup C \subset B \cup C$.
			\item $A \cap B = \emptyset$ e $C \subset A$ implicam que $B \cap C = \emptyset$�.
		\end{enumerate}
	\end{enumerate}
\end{document}