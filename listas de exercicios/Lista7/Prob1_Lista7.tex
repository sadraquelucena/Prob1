\documentclass[20pt,a4paper]{report}
\usepackage[dvips]{graphicx}           
\usepackage[portuges]{babel}           
\usepackage{color}            
\usepackage[latin1]{inputenc}           
\usepackage[normalem]{ulem}
\setlength{\textwidth}{160mm}
\setlength{\textheight}{220mm}
\setlength{\topmargin}{-8pt}
\setlength{\oddsidemargin}{0mm}
\usepackage{amsmath,amsfonts}
\usepackage{enumitem}% http://ctan.org/pkg/enumitem

\begin{document}

  \begin{minipage}[b]{0.1 \linewidth}
     \includegraphics[width=.9cm]{ufs_vertical_positiva.png}
  \end{minipage}\hfill	
  \begin{minipage}[b]{0.85 \linewidth}
	       \textsc{\large Universidade Federal de Sergipe}\\
	       \textsc{\large Departamento de Estat�stica e Ci�ncias Atuariais}\\
	       Disciplina: ESTAT0072 -- Probabilidade I\\
	       Professor: Sadraque E. F. Lucena
\end{minipage}

\vspace{0.8cm}
\begin{center} {\bf Lista de Exerc�cios 7} \end{center}
\vspace{0.8cm}

	\begin{enumerate}[label=7.\arabic*)]
		\item Na tabela abaixo, os n�meros que aparecem s�o probabilidades relacionadas com a ocorr�ncia de $A$, $B$, $A \cap B$ etc. Assim, $P(A) = 0,\!10$, enquanto $P(A \cap B) = 0,\!04$.
		
		\begin{center}
			\begin{tabular}{c|c|c|c}
				\hline
				& $B$ & $B^c$ & Total\\
				\hline
				$A$ & 0,04 & 0,06 & 0,10\\
				$A^c$ & 0,08 & 0,82 & 0,90\\
				\hline
				Total & 0,12 & 0,88 & 1,00\\
				\hline
			\end{tabular}
		\end{center}
		Verifique se $A$ e $B$ s�o independentes.
		
		\item As probabilidades de tr�s motoristas serem capazes de guiar at� em casa com seguran�a, depois de beber, s�o de 1/3, 1/4 e 1/5, respectivamente. Se decidirem guiar at� em casa, depois de beber numa festa,
		\begin{enumerate}
			\item qual a probabilidade de todos os tr�s motoristas sofrerem acidentes?
			\item Qual a probabilidade de pelo menos um dos motoristas guiar at� em casa a salvo?
		\end{enumerate}
		
		\item Lance uma moeda duas vezes e considere os eventos: A: cara no primeiro lan�amento, B: cara no segundo lan�amento e C: faces diferentes nos dois lan�amentos. Mostre que A, B e C s�o mutuamente independentes, mas n�o totalmente independentes. % https://files.cercomp.ufg.br/weby/up/335/o/l4prob.pdf?1409527803
		
		\item Jogue um dado honesto duas vezes. Considere os eventos: A: o resultado do primeiro lan�amento � par, B: o resultado do segundo lan�amento � par e C: a soma dos resultados � par. % https://files.cercomp.ufg.br/weby/up/335/o/l4prob.pdf?1409527803
		\begin{enumerate}
			\item Os eventos s�o mutuamente independentes?
			\item S�o totalmente independentes?
		\end{enumerate}
		
		\item Sejam $A$ e $B$ eventos de um espa�o amostral $\Omega$ tais que $P(A) = \frac{1}{5}$, $P(B) = p$ e $P(A \cup B) = \frac{1}{2}$. Determine o valor de $p$ para que $A$ e $B$ sejam independentes. % https://cesad.ufs.br/ORBI/public/uploadCatalago/10161010102012Probabilidade_e_Estatistica_aula_8.pdf
		
		\item Solicita-se a dois estudantes, Maria e Pedro, que resolvam determinado problema. Eles trabalham na solu��o do mesmo independentemente, e t�m, respectivamente, probabilidade 0,8 e 0,7 de resolv�-lo. % https://cesad.ufs.br/ORBI/public/uploadCatalago/10161010102012Probabilidade_e_Estatistica_aula_8.pdf
		\begin{enumerate}
			\item Qual � a probabilidade de que nenhum deles resolva o problema?
			\item Qual � a probabilidade de o problema ser resolvido?
		\end{enumerate}
	\end{enumerate}
	
	\newpage
	
	\noindent Respostas:
	
	\begin{enumerate}[label=7.\arabic*)]
		\item N�o s�o independentes.
		
		\item \begin{enumerate}
			\item 0,40
			\item 0,60
		\end{enumerate}
		
		\setcounter{enumi}{3} % pula o item 7.3
		
		\item  \begin{enumerate}
			\item Sim
			\item N�o
		\end{enumerate}
		
		\item 3/8
		
		\item \begin{enumerate}
			\item 0,06
			\item 0,94
		\end{enumerate}
	\end{enumerate}
	
\end{document}